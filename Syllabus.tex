\documentclass[11pt,a4paper]{article}
\usepackage[utf8]{inputenc}
\usepackage[T1]{fontenc}
\usepackage[french]{babel}
\usepackage{amsmath,amssymb}
\usepackage[margin=2cm]{geometry}
\usepackage{array}
\usepackage{booktabs}
\pagestyle{empty}
\setlength{\parindent}{0pt}
\setlength{\parskip}{0.4em}

\begin{document}

\begin{center}
\textsc{Université Galatasaray -- Département de Mathématiques}\\[0.1cm]
{\small Année académique 2025--2026}\\[0.3cm]
{\Large\bfseries MAT 205 -- Introduction à la Théorie des Anneaux et des Corps}\\[0.2cm]
{\small\ttfamily https://github.com/onayg/mat205}
\end{center}

\vspace{0.3cm}

\begin{minipage}[t]{0.48\textwidth}
\textbf{Responsable :} Gönenç Onay\\
\texttt{gonay@gsu.edu.tr}
\end{minipage}
\hfill
\begin{minipage}[t]{0.48\textwidth}
\textbf{Horaire :} Jeudi 13h--15h, Vendredi 15h--18h\\
\textbf{Permanence :} Jeudi après le cours ou sur RDV
\end{minipage}

\vspace{0.4cm}

\textbf{Bibliographie.}
D.~Perrin, \textit{Cours d'algèbre}, Ellipses\,;
J.-J.~Risler \& M.~Boyer, \textit{Algèbre pour la Licence 3}, Dunod\,;
M.~Hindry, \textit{Arithmétique}, Calvage \& Mounet (corps finis)\,;
F.~De Marçay, \textit{Groupes, Anneaux, Corps}, polycopié Orsay.
Un résumé de cours sera distribué au fur et à mesure.

\vspace{0.3cm}

\textbf{Contenu.}
\begin{center}
\small
\begin{tabular}{@{}r@{\,:\,}l@{\qquad}r@{\,:\,}l@{}}
\toprule
\textbf{S1} & Anneaux : $\mathbb{Z}$, $\mathbb{Z}/n\mathbb{Z}$, polynômes, matrices &
\textbf{S8} & \textbf{Examen partiel} \\
\textbf{S2} & Unités, diviseurs de zéro, nilpotents, intégrité &
\textbf{S9} & Anneaux principaux, anneaux factoriels \\
\textbf{S3} & Idéaux, quotients, idéaux premiers et maximaux &
\textbf{S10} & Extensions de corps, degré \\
\textbf{S4} & Homomorphismes, théorèmes d'isomorphisme \textbf{(Q1)} &
\textbf{S11} & Extensions algébriques, polynôme minimal \\
\textbf{S5} & Anneaux de polynômes, division, racines &
\textbf{S12} & Corps finis : construction, unicité \textbf{(Q2)} \\
\textbf{S6} & Irréductibilité, critère d'Eisenstein &
\textbf{S13} & Structure des corps finis, $\mathbb{F}_q^\times$ cyclique \\
\textbf{S7} & Révision &
\textbf{S14} & Applications et révision \\
\bottomrule
\end{tabular}
\end{center}

\vspace{0.3cm}

\textbf{Évaluation.}
\begin{center}
\begin{tabular}{@{}lclc@{}}
\toprule
Quiz 1 (S4) & 10\% & Quiz 2 (S12) & 10\% \\
Examen partiel (S8) & 30\% & Examen final & 50\% \\
\bottomrule
\end{tabular}
\end{center}
\smallskip
Note finale : $N = 0{,}1\,Q_1' + 0{,}3\,P + 0{,}1\,Q_2' + 0{,}5\,F$,
où $Q_i' = Q_i$ si présent, $Q_i' = P$ si absence justifiée.
Les quiz n'ont pas de rattrapage ; le coefficient est reporté sur le partiel.

\vspace{0.3cm}

\textbf{Barème.}
\begin{center}
\begin{tabular}{@{}cccccc@{}}
\toprule
AA & BA & BB & CB & CC & FF \\
\midrule
$[90,100]$ & $[80,90[$ & $[70,80[$ & $[60,70[$ & $[50,60[$ & $[0,50[$ \\
\bottomrule
\end{tabular}
\end{center}

\vspace{0.3cm}

\textbf{Modalités.}
La présence est obligatoire (minimum 70\% des séances).
Une feuille de présence sera signée à chaque cours.
L'examen d'excuse du partiel nécessite un justificatif officiel.
L'examen d'excuse du final est le rattrapage de la faculté.
Chaque copie doit mentionner uniquement le numéro d'inscription (anonymat).
Toute fraude ou plagiat est sanctionné conformément au règlement.

\end{document}
